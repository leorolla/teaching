\documentclass[11pt]{article}

\usepackage{amsmath}
\usepackage{amsthm}
\usepackage{amsfonts} 
\usepackage{hyperref}
\usepackage{framed}
\usepackage{color}
\usepackage{amssymb}

\usepackage[a4paper]{geometry}

\usepackage[normalem]{ulem}

\usepackage[utf8]{inputenc}
\usepackage[T1]{fontenc}
\usepackage{lmodern}

\hypersetup{colorlinks=true,urlcolor=blue,linkcolor=black,citecolor=black}

\hyphenpenalty 6000
\tolerance 4000

\newtheorem{theorem}{Theorem}
\newtheorem{proposition}{Proposition}[section]
\newtheorem{lemma}{Lemma}[section]
\newtheorem{remark}{Remark}
\newtheorem{definition}{Definition}[section]
\newtheorem{corollary}{Corollary}[section]
\newtheorem*{ack}{Acknowledgement}

\renewcommand{\epsilon}{\varepsilon}

\newcommand{\abs}[1]{\left\vert #1\right\vert}
\newcommand{\1}{\mathbf{1}}
\newcommand{\inner}[2]{\langle #1, #2 \rangle}
\newcommand{\proj}[1]{P_{\!\!{}_{#1}\,}}

\newcommand{\dd}{{\mathrm{d}}}
\newcommand{\e}{{\mathrm{e}}}
\newcommand{\re}{{\mathrm{re}}}
\newcommand{\0}{\mathbf{0}}

\newcommand{\A}{\mathcal{A}}
\newcommand{\B}{\mathcal{B}}
\newcommand{\cC}{\mathcal{C}}
\newcommand{\D}{\mathcal{D}}

\newcommand{\C}{\mathbb{C}}
\newcommand{\F}{\mathbb{F}}
\newcommand{\cL}{\mathcal{L}}
\newcommand{\R}{\mathbb{R}}
\newcommand{\cS}{\mathcal{S}}
\newcommand{\N}{\mathbb{N}}
\newcommand{\Q}{\mathbb{Q}}
\newcommand{\Z}{\mathbb{Z}}
\newcommand{\T}{\vec{T}} 
\newcommand{\cP}{\mathcal{P}}
\newcommand{\PR}{\mathbb{P}}
\newcommand{\Pb}{\overline{\mathbb{P}}}

\DeclareMathOperator{\mydeg}{\mathsf{deg}}
\DeclareMathOperator{\myspan}{\mathsf{span}}
\DeclareMathOperator{\myrange}{\mathsf{range}}
\DeclareMathOperator{\mynull}{\mathsf{ker}}
\DeclareMathOperator{\myker}{\mathsf{ker}}
\DeclareMathOperator{\mydim}{\mathsf{dim}}
\DeclareMathOperator{\myrank}{\mathsf{rank}}
\DeclareMathOperator{\mytrace}{\mathsf{trace}}
\DeclareMathOperator{\mydet}{\mathsf{det}}
\DeclareMathOperator{\mysgn}{\mathsf{sgn}}
\DeclareMathOperator{\mydiag}{\mathsf{diag}}

\DeclareMathOperator{\mydot}{\boldsymbol{\cdot}}

\renewcommand{\det}{use mydet}

\newcommand{\va}{\vec{a}}
\newcommand{\vb}{\vec{b}}
\newcommand{\vc}{\vec{c}}
\newcommand{\ve}{\vec{e}}
\newcommand{\vx}{\vec{x}}
\newcommand{\vy}{\vec{y}}
\newcommand{\vr}{\vec{r}}
\newcommand{\vu}{\vec{u}}
\newcommand{\vv}{\vec{v}}
\newcommand{\vw}{\vec{w}}

\pagestyle{empty}

\parindent 0pt
\parskip .4em

\renewcommand{\leq}{\leqslant}
\renewcommand{\geq}{\geqslant}

\usepackage{graphicx}
\usepackage{hyperref}

\usepackage{embedall}


\begin{document}

\title{PLU Factorization}
\author{Leonardo T. Rolla}
\maketitle

{\let\thefootnote\relax
\footnotetext{\copyright 2019-\the\year\ Leonardo T. Rolla
\href{http://creativecommons.org/licenses/by-sa/3.0/}
{\includegraphics[height=1.0em]{by-sa.pdf}}. This typeset file has the source code embedded in it. If you re-use part of this code, you are kindly requested --if possible-- to convey the source code along with or embedded in the typeset file, and to keep this request.}}

%\subsection*{Theory}

LU Factorization is like performing row reduction on an invertible matrix $A$ until it becomes an upper triangular echelon form $U$.
The difference is that we keep track of the operations in a smart way.
The operations are encoded in a lower triangular matrix $L$, so that $A=LU$.
With this factorization, the system $Ax=b$ can later on be solved for any vector $b$.

During the pure row reduction procedure, sometimes row exchange is needed if we have a zero entry on the diagonal.
Even when the entry is not zero but small, this is still very bad as it will imply adding a very large multiple of this row to the ones below it.
To minimize the impact of rounding off error, a simple technique is that of \emph{partial pivoting}.
It consists in always performing a row exchange, taking as pivot the largest candidate in the current column.
A permutation matrix $P$ will keep track of these row exchanges, resulting in the factorization $PA = LU$.
To solve $Ax=b$, we permute $b$, solve $Lz=Pb$ for $z$ using forward substitution, then solve $Ux=z$ for $x$ using backward substitution.

To explain the factorization we proceed as follows.
Start with $P_0 = I, L_0 = I, U_0=A$.
At all steps, we have \[P_k A = L_k U_k.\]
Here $P_k$ is always a permutation matrix, $L_k$ is always lower triangular, and $U_k$ will be upper triangular after the last step.
The next step always consists in applying row replacement $E$ or row exchange $Q$ to $U_k$ towards its row reduction, and compensate accordingly to keep the above equality. At each step, we have an elementary matrix $E$ or $Q$, and apply either
\[
\begin{cases}
P_{k+1} = P_k \\
L_{k+1} = L_k E^{-1} \\
U_{k+1} = E U_k
\end{cases}
\quad
\text{or}
\quad
\begin{cases}
P_{k+1} = Q P_k \\
L_{k+1} = Q L_k Q \\
U_{k+1} = Q U_k
\end{cases}
.
\]
Note that $Q^{-1}=Q$.
Moreover, when the procedure is done in the right order, $Q L_k Q$ swaps two lower rows of $L_k$ without moving the diagonal terms.
The reader can check this in the example below.

\section*{Worked example}

Start with
\[
A=
\begin{bmatrix}
  1	& -1	&  1	&  2 \\
 -2	&  1	&  1	&  1 \\
  2	& -1	&  2	&  3 \\
 -4	&  1	&  0	&  2 \\
\end{bmatrix}
,
\]
so we have the trivial equality
\[
\begin{bmatrix}
  1	&  0	&  0	&  0 \\
  0	&  1	&  0	&  0 \\
  0	&  0	&  1	&  0 \\
  0	&  0	&  0	&  1 \\
\end{bmatrix}
A
=
\begin{bmatrix}
  1	&  0	&  0	&  0 \\
  0	&  1	&  0	&  0 \\
  0	&  0	&  1	&  0 \\
  0	&  0	&  0	&  1 \\
\end{bmatrix}
\begin{bmatrix}
  1	& -1	&  1	&  2 \\
 -2	&  1	&  1	&  1 \\
  2	& -1	&  2	&  3 \\
 -4	&  1	&  0	&  2 \\
\end{bmatrix}
.
\]

To have $-4$ as the pivot, we apply the permutation
\[
Q = 
\begin{bmatrix}
  0	&  0	&  0	&  1 \\
  0	&  1	&  0	&  0 \\
  0	&  0	&  1	&  0 \\
  1	&  0	&  0	&  0 \\
\end{bmatrix}
.
\]
So we take $P' = Q I$, $L' = Q I Q = I$, $U' = QU$.
The equality then becomes
\[
\begin{bmatrix}
  0	&  0	&  0	&  1 \\
  0	&  1	&  0	&  0 \\
  0	&  0	&  1	&  0 \\
  1	&  0	&  0	&  0 \\
\end{bmatrix}
A
=
\begin{bmatrix}
  1	&  0	&  0	&  0 \\
  0	&  1	&  0	&  0 \\
  0	&  0	&  1	&  0 \\
  0	&  0	&  0	&  1 \\
\end{bmatrix}
\begin{bmatrix}
 -4	&  1	&  0	&  2 \\
 -2	&  1	&  1	&  1 \\
  2	& -1	&  2	&  3 \\
  1	& -1	&  1	&  2 \\
\end{bmatrix}
.
\]

Now apply three row replacements (which we do at once to save space), using
\[
E =
\begin{bmatrix}
  1	&  0	&  0	&  0 \\
 -1/2	&  1	&  0	&  0 \\
  1/2	&  0	&  1	&  0 \\
  1/4	&  0	&  0	&  1 \\
\end{bmatrix}
.
\]
By taking $L' = LE^{-1}$ and $U'=EU$, the previous equality becomes
\[
\begin{bmatrix}
  0	&  0	&  0	&  1 \\
  0	&  1	&  0	&  0 \\
  0	&  0	&  1	&  0 \\
  1	&  0	&  0	&  0 \\
\end{bmatrix}
A
=
\begin{bmatrix}
  1	&  0	&  0	&  0 \\
  1/2	&  1	&  0	&  0 \\
 -1/2	&  0	&  1	&  0 \\
 -1/4	&  0	&  0	&  1 \\
\end{bmatrix}
\begin{bmatrix}
 -4	&  1	&  0	&  2 \\
  0	&  1/2	&  1	&  0 \\
  0	& -1/2	&  2	&  4 \\
  0	& -3/4	&  1	&  5/2 \\
\end{bmatrix}
.
\]

We now want $-3/4$ as the pivot, so we apply
\[
Q =
\begin{bmatrix}
  1	&  0	&  0	&  0 \\
  0	&  0	&  0	&  1 \\
  0	&  0	&  1	&  0 \\
  0	&  1	&  0	&  0 \\
\end{bmatrix}
.
\]
As before, taking $P' = Q P$, $L' = Q L Q$ and $U' = QU$, the equality becomes
\[
\begin{bmatrix}
  0	&  0	&  0	&  1 \\
  1	&  0	&  0	&  0 \\
  0	&  0	&  1	&  0 \\
  0	&  1	&  0	&  0 \\
\end{bmatrix}
A
=
\begin{bmatrix}
  1	&  0	&  0	&  0 \\
 -1/4	&  1	&  0	&  0 \\
 -1/2	&  0	&  1	&  0 \\
  1/2	&  0	&  0	&  1 \\
\end{bmatrix}
\begin{bmatrix}
 -4	&  1	&  0	&  2 \\
  0	& -3/4	&  1	&  5/2 \\
  0	& -1/2	&  2	&  4 \\
  0	&  1/2	&  1	&  0 \\
\end{bmatrix}
.
\]

We now do two row replacements at once, using
\[
E =
\begin{bmatrix}
  1	&  0	&  0	&  0 \\
  0	&  1	&  0	&  0 \\
  0	& -2/3	&  1	&  0 \\
  0	&  2/3	&  0	&  1 \\
\end{bmatrix}
.
\]
After this operation, taking $L' = LE^{-1}$ and $U'=EU$ the equality becomes:
\[
\begin{bmatrix}
  0	&  0	&  0	&  1 \\
  1	&  0	&  0	&  0 \\
  0	&  0	&  1	&  0 \\
  0	&  1	&  0	&  0 \\
\end{bmatrix}
A
=
\begin{bmatrix}
  1	&  0	&  0	&  0 \\
 -1/4	&  1	&  0	&  0 \\
 -1/2	&  2/3	&  1	&  0 \\
  1/2	& -2/3	&  0	&  1 \\
\end{bmatrix}
\begin{bmatrix}
 -4	&  1	&  0	&  2 \\
  0	& -3/4	&  1	&  5/2 \\
  0	&  0	&  4/3	&  7/3 \\
  0	&  0	&  5/3	&  5/3 \\
\end{bmatrix}
.
\]

We now want $5/3$ as the pivot, so we apply
\[
Q =
\begin{bmatrix}
  1	&  0	&  0	&  0 \\
  0	&  1	&  0	&  0 \\
  0	&  0	&  0	&  1 \\
  0	&  0	&  1	&  0 \\
\end{bmatrix}
.
\]

Again, taking $P' = Q P$, $L' = Q L Q$ and $U' = QU$, the equality becomes
%Again, permuting the corresponding rows of $P$ and $U$, and permuting the corresponding rows as well as corresponding columns of $L$, we get
\[
\begin{bmatrix}
  0	&  0	&  0	&  1 \\
  1	&  0	&  0	&  0 \\
  0	&  1	&  0	&  0 \\
  0	&  0	&  1	&  0 \\
\end{bmatrix}
A
=
\begin{bmatrix}
  1	&  0	&  0	&  0 \\
 -1/4	&  1	&  0	&  0 \\
  1/2	& -2/3	&  1	&  0 \\
 -1/2	&  2/3	&  0	&  1 \\
\end{bmatrix}
\begin{bmatrix}
 -4	&  1	&  0	&  2 \\
  0	& -3/4	&  1	&  5/2 \\
  0	&  0	&  5/3	&  5/3 \\
  0	&  0	&  4/3	&  7/3 \\
\end{bmatrix}
.
\]

The last step is the row replacement given by
\[
E =
\begin{bmatrix}
  1	&  0	&  0	&  0 \\
  0	&  1	&  0	&  0 \\
  0	&  0	&  1	&  0 \\
  0	&  0	& -4/5	&  1 \\
\end{bmatrix}
.
\]
This finally yields the factorization
\[
\begin{bmatrix}
  0	&  0	&  0	&  1 \\
  1	&  0	&  0	&  0 \\
  0	&  1	&  0	&  0 \\
  0	&  0	&  1	&  0 \\
\end{bmatrix}
A
=
\begin{bmatrix}
  1	&  0	&  0	&  0 \\
 -1/4	&  1	&  0	&  0 \\
  1/2	& -2/3	&  1	&  0 \\
 -1/2	&  2/3	&  4/5	&  1 \\
\end{bmatrix}
\begin{bmatrix}
 -4	&  1	&  0	&  2 \\
  0	& -3/4	&  1	&  5/2 \\
  0	&  0	&  5/3	&  5/3 \\
  0	&  0	&  0	&  1 \\
\end{bmatrix}
.
\]

\section*{Compact representation}

Since the diagonal of $L$ only has $1$, the factorization can be encoded in an $n \times n$ matrix plus a permutation vector:

\[
\begin{array}{|r|rrr|rrrr|}
\hline
  1	& \phantom{-0.000}&\phantom{-0.000}&\phantom{-0.000}&  1.000	& -1.000	&  1.000	&  2.000 \\
  2	& 		& 		&		& -2.000	&  1.000	&  1.000	&  1.000 \\
  3	& 		& 		&  		&  2.000	& -1.000	&  2.000	&  3.000 \\
  4	& 		& 		& 		& -4.000	&  1.000	&  0.000	&  2.000 \\
\hline
\end{array}
\]

\[
\begin{array}{|r|rrr|rrrr|}
\hline
  4	& \phantom{-0.000}&\phantom{-0.000}&\phantom{-0.000}& -4.000	&  1.000	&  0.000	&  2.000 \\
  2	& 		&  		&		& -2.000	&  1.000	&  1.000	&  1.000 \\
  3	& 		& 		&  		&  2.000	& -1.000	&  2.000	&  3.000 \\
  1	& 	 	& 		& 		&  1.000	& -1.000	&  1.000	&  2.000 \\
\hline
\end{array}
\]

\[
\begin{array}{|r|rrr|rrrr|}
\hline
  4	& \phantom{-0.000}&\phantom{-0.000}&\phantom{-0.000}& -4.000	&  1.000	&  0.000	&  2.000 \\
  2	&  0.500	&  		&		& 		&  0.500	&  1.000	&  1.000 \\
  3	& -0.500	& 		&  		& 		& -0.500	&  2.000	&  3.000 \\
  1	& -0.250 	& 		& 		& 		& -0.750	&  1.000	&  2.500 \\
\hline
\end{array}
\]

\[
\begin{array}{|r|rrr|rrrr|}
\hline
  4	& \phantom{-0.000}&\phantom{-0.000}&\phantom{-0.000}& -4.000	&  1.000	&  0.000	&  2.000 \\
  1	& -0.250	&  		&		& 		& -0.750	&  1.000	&  2.500 \\
  3	& -0.500	& 		&  		& 		& -0.500	&  2.000	&  3.000 \\
  2	&  0.500 	& 		& 		& 		&  0.500	&  1.000	&  1.000 \\
\hline
\end{array}
\]

\[
\begin{array}{|r|rrr|rrrr|}
\hline
  4	& \phantom{-0.000}&\phantom{-0.000}&\phantom{-0.000}& -4.000	&  1.000	&  0.000	&  2.000 \\
  1	& -0.250	&  		&		& 		& -0.750	&  1.000	&  2.500 \\
  3	& -0.500	&  0.667	&  		& 		& 		&  1.333	&  2.333 \\
  2	&  0.500 	& -0.667	& 		& 		& 		&  1.667	&  1.667 \\
\hline
\end{array}
\]

\[
\begin{array}{|r|rrr|rrrr|}
\hline
  4	& \phantom{-0.000}&\phantom{-0.000}&\phantom{-0.000}& -4.000	&  1.000	&  0.000	&  2.000 \\
  1	& -0.250	&  		&		& 		& -0.750	&  1.000	&  2.500 \\
  2	&  0.500 	& -0.667	& 		& 		& 		&  1.667	&  1.667 \\
  3	& -0.500	&  0.667	& 		& 		& 		&  1.333	&  2.333 \\
\hline
\end{array}
\]

\[
\begin{array}{|r|rrr|rrrr|}
\hline
  4	& \phantom{-0.000}&\phantom{-0.000}&\phantom{-0.000}& -4.000	&  1.000	&  0.000	&  2.000 \\
  1	& -0.250	&  		&		& 		& -0.750	&  1.000	&  2.500 \\
  2	&  0.500 	& -0.667	& 		& 		& 		&  1.667	&  1.667 \\
  3	& -0.500	&  0.667	&  0.800	& 		& 		&  		&  1.000 \\
\hline
\end{array}
\]

\pagebreak[4]

\section*{Matrix manipulation software}

Below is the output of \emph{Octave} when we ask it for the PLU factorization of $A$:

\begin{verbatim}
octave:1> A = [ 1 -1 1 2 ; -2 1 1 1 ; 2 -1 2 3 ; -4 1 0 2 ]
A =

   1  -1   1   2
  -2   1   1   1
   2  -1   2   3
  -4   1   0   2

octave:2> [L, U, P] = lu(A)
L =

   1.00000   0.00000   0.00000   0.00000
  -0.25000   1.00000   0.00000   0.00000
   0.50000  -0.66667   1.00000   0.00000
  -0.50000   0.66667   0.80000   1.00000

U =

  -4.00000   1.00000   0.00000   2.00000
   0.00000  -0.75000   1.00000   2.50000
   0.00000   0.00000   1.66667   1.66667
   0.00000   0.00000   0.00000   1.00000

P =

   0   0   0   1
   1   0   0   0
   0   1   0   0
   0   0   1   0

octave:3> _
\end{verbatim}

The computations in the previous pages show that \emph{Octave} got it right indeed ;-)

\pagebreak[4]

\section*{Efficiency of LU factorization}

Finding the LU factorization is as fast as finding $A_\e$ by row reduction, and at least 3 times faster than finding $A^{-1}$.
Suppose you have found $L$, $U$ and $A^{-1}$, and you want to solve $Ax=b$ for many different vectors $b$.
Solving $Ax=b$ by $Lz=b$ and $Ux=z$ is at least as fast as computing the product $A^{-1}b$, and the result is more accurate.

If the matrix has many zeros, using LU is much better and faster than using $A^{-1}$.

{\small
\begin{verbatim}
octave:23> A
A =

   2.00000   1.00000   0.00000   0.00000   0.00000   0.00000
   1.00000   3.00000   1.00000   0.00000   0.00000   0.00000
   0.00000   1.00000   3.00000   1.00000   0.00000   0.00000
   0.00000   0.00000   1.00000   3.00000   1.00000   0.00000
   0.00000   0.00000   0.00000   1.00000   3.00000   1.00000
   0.00000   0.00000   0.00000   0.00000   1.00000   2.00000

octave:24> [L,U] = lu(A)
L =

   1.00000   0.00000   0.00000   0.00000   0.00000   0.00000
   0.50000   1.00000   0.00000   0.00000   0.00000   0.00000
   0.00000   0.40000   1.00000   0.00000   0.00000   0.00000
   0.00000   0.00000   0.38462   1.00000   0.00000   0.00000
   0.00000   0.00000   0.00000   0.38235   1.00000   0.00000
   0.00000   0.00000   0.00000   0.00000   0.38202   1.00000

U =

   2.00000   1.00000   0.00000   0.00000   0.00000   0.00000
   0.00000   2.50000   1.00000   0.00000   0.00000   0.00000
   0.00000   0.00000   2.60000   1.00000   0.00000   0.00000
   0.00000   0.00000   0.00000   2.61538   1.00000   0.00000
   0.00000   0.00000   0.00000   0.00000   2.61765   1.00000
   0.00000   0.00000   0.00000   0.00000   0.00000   1.61798

octave:25> inv(A)
ans =

   0.61806  -0.23611   0.09028  -0.03472   0.01389  -0.00694
  -0.23611   0.47222  -0.18056   0.06944  -0.02778   0.01389
   0.09028  -0.18056   0.45139  -0.17361   0.06944  -0.03472
  -0.03472   0.06944  -0.17361   0.45139  -0.18056   0.09028
   0.01389  -0.02778   0.06944  -0.18056   0.47222  -0.23611
  -0.00694   0.01389  -0.03472   0.09028  -0.23611   0.61806

octave:26> _
\end{verbatim}
}

\end{document}

