\documentclass[11pt]{article}


\usepackage{amsmath}
\usepackage{amsthm}
\usepackage{amsfonts} 
\usepackage{hyperref}
\usepackage{framed}
\usepackage{color}
\usepackage{amssymb}

\usepackage[normalem]{ulem}

\usepackage[utf8]{inputenc}
\usepackage[T1]{fontenc}
\usepackage{lmodern}

\hypersetup{colorlinks=true,urlcolor=blue,linkcolor=black,citecolor=black}

\hyphenpenalty 6000
\tolerance 4000

\newtheorem{theorem}{Theorem}
\newtheorem{proposition}{Proposition}[section]
\newtheorem{lemma}{Lemma}[section]
\newtheorem{remark}{Remark}
\newtheorem{definition}{Definition}[section]
\newtheorem{corollary}{Corollary}[section]
\newtheorem*{ack}{Acknowledgement}

\renewcommand{\epsilon}{\varepsilon}

\newcommand{\abs}[1]{\left\vert #1\right\vert}
\newcommand{\1}{\mathbf{1}}
\newcommand{\inner}[2]{\langle #1, #2 \rangle}
\newcommand{\proj}[1]{P_{\!\!{}_{#1}}}

\newcommand{\dd}{{\mathrm{d}}}
\newcommand{\e}{{\mathrm{e}}}
\newcommand{\re}{{\mathrm{re}}}
\newcommand{\0}{\mathbf{0}}

\newcommand{\A}{\mathcal{A}}
\newcommand{\B}{\mathcal{B}}
\newcommand{\cC}{\mathcal{C}}
\newcommand{\D}{\mathcal{D}}

\newcommand{\C}{\mathbb{C}}
\newcommand{\F}{\mathbb{F}}
\newcommand{\cL}{\mathcal{L}}
\newcommand{\R}{\mathbb{R}}
\newcommand{\cS}{\mathcal{S}}
\newcommand{\N}{\mathbb{N}}
\newcommand{\Q}{\mathbb{Q}}
\newcommand{\Z}{\mathbb{Z}}
\newcommand{\T}{\vec{T}} 
\newcommand{\cP}{\mathcal{P}}
\newcommand{\PR}{\mathbb{P}}
\newcommand{\Pb}{\overline{\mathbb{P}}}

\DeclareMathOperator{\mydeg}{\mathsf{deg}}
\DeclareMathOperator{\myspan}{\mathsf{span}}
\DeclareMathOperator{\myrange}{\mathsf{range}}
\DeclareMathOperator{\mynull}{\mathsf{ker}}
\DeclareMathOperator{\myker}{\mathsf{ker}}
\DeclareMathOperator{\mydim}{\mathsf{dim}}
\DeclareMathOperator{\myrank}{\mathsf{rank}}
\DeclareMathOperator{\mytrace}{\mathsf{trace}}
\DeclareMathOperator{\mydet}{\mathsf{det}}
\DeclareMathOperator{\mysgn}{\mathsf{sgn}}
\DeclareMathOperator{\mydiag}{\mathsf{diag}}

\DeclareMathOperator{\mydot}{\boldsymbol{\cdot}}


\renewcommand{\det}{use mydet}

\newcommand{\va}{\vec{a}}
\newcommand{\vb}{\vec{b}}
\newcommand{\vc}{\vec{c}}
\newcommand{\ve}{\vec{e}}
\newcommand{\vx}{\vec{x}}
\newcommand{\vy}{\vec{y}}
\newcommand{\vr}{\vec{r}}
\newcommand{\vu}{\vec{u}}
\newcommand{\vv}{\vec{v}}
\newcommand{\vw}{\vec{w}}

\newcommand{\abba}{{\tiny \begin{bmatrix} a \ {-\!\!\;b} \\ b \ \ \ a \end{bmatrix}}}

\pagestyle{empty}

\parindent 0pt
\parskip .4em

\renewcommand{\leq}{\leqslant}
\renewcommand{\geq}{\geqslant}


\usepackage[a4paper]{geometry}

\theoremstyle{definition}
\newtheorem{example}{Example}
\usepackage{graphicx}
\usepackage{hyperref}


\usepackage{embedall}


\begin{document}


\section*{Methods to find a Jordan basis}

\begin{center}
\Large
\textit{Note: we use $ (a,b,c) $ to denote the column vector $ [ \ a \ b \ c \ ]^T $.}
\end{center}

{\let\thefootnote\relax
\footnotetext{\copyright \the\year\ Leonardo T. Rolla
\href{http://creativecommons.org/licenses/by-sa/3.0/}
{\includegraphics[height=1.0em]{by-sa.pdf}}. This typeset file has the source code embedded in it. If you re-use part of this code, you are kindly requested --if possible-- to convey the source code along with or embedded in the typeset file, and to keep this request.}}

\subsection*{Quick and Dirty methods}

\begin{itemize}
%\item
%Find all the eigenvalues.
\item
General method.
For each eigenvalue $\lambda$:
\begin{itemize}
\item
Find the eigenspace $E(\lambda,T)$ by solving $Tu = \lambda u$.
\item
Find a basis $\A$ to the eigenspace $E(\lambda,T)$.
\item
For each $v$ in $\A$:
\begin{itemize}
\item
Find one $v'$ which solves $Tv' = \lambda v' + v$, if possible.
\item
Find one $v''$ which solves $Tv'' = \lambda v'' + v'$, if possible.
\item
Find $v'''$, $v''''$, etc., until the equation has no solutions.
\end{itemize}
\end{itemize}
The result is always an L.I. family, but may not be spanning.
\item
Method indicated for the case of a unique $\lambda$:
\begin{itemize}
\item
Pick a $v$ at random on $ G(\lambda,T) $, write $u=v$.
\item
Let $u' = Tu - \lambda u$, $u'' = Tu' - \lambda u'$, $u'''= Tu'' - \lambda u''$, etc., until it gives $\0$.
\item
If fewer than $n$ vectors have been found, find $v',v'',v''',\dots$ as above.
\item
Pick random $v$ outside the span of previous vectors, and repeat the process.
\end{itemize}
The result is always a spanning family, but may not be L.I.
%\item
%Check that the previous steps:
%\begin{enumerate}
%\item
%produce exactly $n$ vectors;
%\item
%the vectors are linearly independent.
%\end{enumerate}
%If these conditions are met, the obtained family $\B$ forms a Jordan basis.
%%\begin{proof}
%%%For having $n$ L.I.\ vectors, 
%%$\B$ is a basis, and $[T]_\B$ has the desired form because $Tv' = \lambda v' + v$ etc.
%%\end{proof}
%\par
%If not, \textbf{the method has failed}, and it is better to use the Guaranteed Method.
\end{itemize}

\subsubsection*{Comments}

%Both methods will typically fail
%If Jordan blocks of an eigenvalue $ \lambda $ have different sizes, this will typically be the case.

%Typically,

Typically, these methods fail if and only if there is an eigenvalue $ \lambda $ whose Jordan blocks have different sizes.
Exceptions in both directions are unlikely or impossible.

The first method will fail if the basis $\A$ does not have vectors $v$ that belong to $\myrange (T-\lambda I)^k$ with $k$ as large as possible.
Then the chain $v,v',v'',\dots$ will not be long enough.

An example is $T(x,y,z)=(x,y+z,z)$, so $ \lambda=1 $ and $\A = \{ (1,1,0), (1,2,0) \}$ for $E(1,T)$.
The basis $\A$ does not have a vector in $\myrange(T-I)$.
So the equation $Tv' = v' + v$ has no solutions, and the method falls short of producing 3 vectors.

The second method will fail if the threads
$ u_1 \mapsto u_1' \mapsto \cdots $,
$ u_2 \mapsto u_2' \mapsto \cdots $, etc become linearly dependent instead of reaching $ \0 $.

An example is $T(x,y,z)=(x,y+z,z)$ with $ u_1 = (1,2,3) $, $ u_1'=(0,3,0) $, $ u_1''=\0 $ and $ u_2=(1,1,1), u_2'=(0,1,0), u_2''=\0 $, so $ u_2' $ is a multiple of $ u_1' $.

%For simplicity, we apply the multiple eigenvalue case to $T(x,y,z)=(x,y+z,z)$ even though it has a single eigenvalue.

%There are also examples where the method produces $n$ vectors but they are not linearly independent.

\clearpage

\subsection*{Guaranteed method}

\begin{itemize}
\item
Find all the eigenvalues.
\item
For each eigenvalue $\lambda$:
\begin{itemize}
\item
Let $N = T-\lambda I$.
\item
Compute $N^2, N^3, \dots, N^n$.
\item
Find the generalized eigenspace $G=G(\lambda,T)$ of solutions $u$ to $N^n u = \0$.
\item
Find a temporary basis for $G$.
\item
Let $U_0 = G$, $U_n = \{\0\}$ and $\B_n = \emptyset$. Then $\B_n$ is a Jordan basis for $U_n$.
\item
For $k=n-1,\dots,1,0$:
\begin{itemize}
\item
Find $U_k = \myrange (N_{|_G})^k$ by applying $N^k$ to the temporary basis of $G$.
\item
From the previous step we have a Jordan basis $\B_{k+1}$ to $T_{|_{U_{k+1}}}$
\\
given by
$N^{d_1}v_1,\dots,N^2 v_1, N v_1, v_1, \dots, N^{d_m}v_m,\dots,N^2 v_m, N v_m, v_m$,
\\
with the property that $N^{d_j+1} v_j = \0$ for all $j$.
\item
For $j=1,\dots,m$, find one $u_j$ such that $Nu_j = v_j$.
\par
Let $\tilde{\B}_k = N^{d_1}v_1,\dots,N^2 v_1, N v_1, v_1, u_1 \dots, N^{d_m}v_m,\dots,N^2 v_m, N v_m, v_m, u_m$
\par
Then $\tilde{\B}_k$ is a Jordan basis for $T_{|_{\myspan\tilde\B_k}}$.
\item
Find $\A_k$ be such that $\tilde{\B}_k \cup \A_k$ is a basis of $U_k$.
\item
For each $w \in \A_k$:
\begin{itemize}
\item
Find $x \in \myspan \tilde{\B}_k$ such that $Nx=Nw$.
\item
Let $u=w-x$, so $Nu = \0$.
\end{itemize}
\item
Let $\tilde{\A}_k$ be the set of vectors obtained above, so $\# \tilde\A_k = \#\A_k$.
\item
Let $\B_k = \tilde{\B}_k, \tilde{\A}_k$.
Then $\B_k$ is a Jordan basis for $T_{|_{U_k}}$.
\end{itemize}
\item
In the end, $\B_0$ is a Jordan basis for $T_{|_G}$.
\end{itemize}
\item
Recollecting the Jordan bases for each $T_{|_{G(\lambda,T)}}$ produces a Jordan basis for $T$.
\end{itemize}


\subsection*{Comment}

This method is guaranteed because is based on the proof of existence of Jordan bases found in Axler's Linear Algebra Done Right.

In the previous example, $U_1 = \myspan(0,11,0)$.
We can take $\A_1 = \{(0,-7,0)\}$, then $\tilde{\A_1}=\A_1$ and $\B_1=\A_1$.
By solving $Nu=(0,-7,0)$ we can take $u=(5,8,-7)$ and $\tilde{\B}_0 = \{(0,-7,0),(5,8,-7)\}$.
In order to extend $\tilde{\B}_0$ to a basis of $U_0 = \C^3$ we can take $\A_0 = \{(3,-2,7)\}$.
For $w=(3,-2,7)$, we have $Nw=(0,7,0)$.
Solving for $Nx = Nw$, the only solution $x \in \myspan(5,8,-7)$ is $x = (-5,-8,7)$, hence $u=w-x=(8,6,0)$ and $\tilde{\A}_0 = \{(8,6,0)\}$.
Finally, $\B_0 = \tilde{\B}_0 \cup \tilde{\A}_0 = \{ (0,-7,0),(5,8,-7),(8,6,0) \}$ is a Jordan basis.

%\subsubsection*{Example}
% $T(x,y,z)=(x,y+z,z)$.


\clearpage

\section*{Examples}

\begin{example}
\[
[T] = A =
\begin{bmatrix}
-4 & 9 \\
-4 & 8
\end{bmatrix}
.
\]
\\
\end{example}

First with the Quick and Dirty method.
\\

Compute eigenvalues: $ \mydet(A-\lambda I)=0 $... get $\lambda=2$.

Pick a random vector: $ u = (5,3) $.

Take $ u' = Au - 2u $.
Multiplying... $u' = (-3,-2)$.

Quick and Dirty method succeeded!

We already know what the Jordan form is and how to write the basis.
Let's double-check:
\[
Q=
\begin{bmatrix}
-3 & 5 \\
-2 & 3
\end{bmatrix}
\
\Longrightarrow
\
Q^{-1}AQ = 
\begin{bmatrix}
2 & 1 \\
0 & 2
\end{bmatrix}
.
\]
\\

Let's see with the Guaranteed Method.
\\

Compute eigenvalues: $ \mydet(A-\lambda I)=0 $... get $\lambda=2$.

Take $N=A-2I$.
Multiplying... $N^2 = \0$, so $G(2,T)=\C^2$, take the canonical basis.

We now compute $U_1$.

Multiplying... $y = Ne_1 = (-6,-4)$ and $y' = Ne_2 = (9,6)$.

Perform row reduction on $[y,y']$... we see that $\B_1=\{y\}$ is a basis for $U_1 = \myrange N$.

We now compute $U_2$.

Multiplying... $Ny = \0$, so $U_2 = \{\0\}$.

We now build the basis from top down:

For $k=2$, $\B_2 = \emptyset$.

For $k = 1$:

$U_1$ is one-dimensional, so take $\B_1 = \{y\}$.

For $k=0$:

Solving $Nx = y$ we get a solution $x=(4,2)$.
So ${\B_0}=\{y,x\}$.

We already know what the Jordan form is and how to write the basis.
Let's double-check:
\[
Q=
\begin{bmatrix}
-6 & 4 \\
-4 & 2
\end{bmatrix}
\
\Longrightarrow
\
Q^{-1}AQ = 
\begin{bmatrix}
2 & 1 \\
0 & 2
\end{bmatrix}
.
\]


\clearpage

\begin{example}
\[
[T]=
A=
\begin{bmatrix}
-2 & 2 & 1 \\
-7 & 4 & 2 \\
5 & 0 & 0
\end{bmatrix}
.
\]
\\
\end{example}

First with the Quick and Dirty method.
\\

Compute eigenvalues: $ \mydet(A-\lambda I)=0 $... get $\lambda=1$ and $0$.

For $\lambda=0$:

Solve $Ax=\0$... get $u=(0,1,-2)$.

Solve $Ax=u$...
no solutions (echelon form has a pivot at the last column).

For $\lambda=1$:

Solve $Ax=x$... get $ v=(1,-1,5) $.

Solve $Ax=x+v$... get $ v'= (1,2,0) $.

Solve $Ax=x+v'$...
no solutions (echelon form has a pivot at the last column).

Vectors $v,v' \in G(1,T)$ are L.I.\ because they belong to the same thread $v' \stackrel{N}{\mapsto} v \stackrel{N}{\mapsto} \0$.
Vectors $u,v,v'$ are L.I.\ because $u$ belongs to $G(0,T)$.

Quick and Dirty method succeeded!

We already know what the Jordan form is and how to write the basis.
Let's double-check:
\[
Q=
\begin{bmatrix}
0 & 1 & 1 \\
1 & -1 & 2 \\
-2 & 5 & 0
\end{bmatrix}
\
\Longrightarrow
\
Q^{-1}AQ = 
\left[
\begin{array}{r|rr}
0 & 0 & 0 \\
\hline
0 & 1 & 1 \\
0 & 0 & 1
\end{array}
\right]
.
\]
\\
\\

Let's see with the Guaranteed Method.
\\

Compute eigenvalues: $ \mydet(A-\lambda I)=0 $... get $\lambda=1$ and $0$.

For $\lambda=1$:

Take $N=A-I$.
Multiplying...
\[
N^3 =
\begin{bmatrix}
0 & 0 & 0 \\
-10 & 5 & 3 \\
20 & -10 & -6
\end{bmatrix}
\]

Solving $N^3 x = \0$... $U_0 = G(1,T)=\myspan \{y,y'\}$ with $y=(6,0,20)$ and $y'=(5,10,0)$.

We now compute $U_1$.

Multiplying... $Ny = (2,-2,10)$ and $Ny' = (5,-5,25)$.

Doing row reduction on $ [Ny, Ny'] $... we get only one pivot, and it is at the first column.

Hence, $z = (2,-2,10)$ is a basis for $U_1$.

We now compute $U_2$.

Multiplying... $Nz = \0$, so $U_2 = U_3 = \{\0\}$.

We now build the basis from top down:

For $k=3$, $\B_3 = \emptyset$.

For $k=2$, $\B_2 = \emptyset$.

For $k=1$:

We can take $\A_1=\{w\}$ with $w=z$.

No need to multiply since we know $Nw \in U_2 = \{\0\}$, so we take $\B_1=\tilde{\A}_1=\{(2,-2,10)\}$.

For $k=0$:

Solving $Nx=z$... get $v=(2,4,0)$ as solution.

So we take $\tilde{\B}_0 = \{z,v\}$.

Since $\mydim U_0 = 2$, we take $\A_0 = \emptyset$, $\tilde{\A}_0 = \emptyset$, and $\B_0=\{(2,-2,10),(2,4,0)\}$.

For $\lambda=0$:

Take $N=A$.
Multiplying...
\[
N^3
=
\begin{bmatrix}
-8 & 6 & 3 \\
-1 & 0 & 0 \\
-25 & 20 & 10
\end{bmatrix}
.
\]
Solving $N^3 x = \0$... $U_0=G(0,T)=\myspan(x)$ with $x=(0,1,-2)$.

Since $G(0,T)$ is one-dimensional, the guaranteed method will not do much here.

Compute range by multiplying...
$N x = \0$.

Hence $ U_3 = U_2 = U_1 = \{\0\} $ and $\B_3 = \B_2 = \B_1 = \emptyset$.

So $\tilde{\B_0}=\emptyset$ as a basis for $U_0$ we can take ${\A}_0=\{w\}$ with $w=(0,1,-2)$.

Multiplying... $Nw=\0$, so we take $x=\0$ and $u=w$.
So $\B_0 = \{(0,1,-2)\}$.

Finished.

We already know what the Jordan form is and how to write the basis.
Let's double-check:
\[
Q=
\begin{bmatrix}
2 & 2 & 0 \\
-2 & 4 & 1 \\
10 & 0 & -2
\end{bmatrix}
\
\Longrightarrow
\
Q^{-1}AQ = 
\left[
\begin{array}{rr|r}
1 & 1 & 0 \\
0 & 1 & 0 \\
\hline
0 & 0 & 0
\end{array}
\right]
.
\]


\clearpage

\begin{example}
\[
[T] = A =
\begin{bmatrix}
-1 & -1 & 3 \\
0 & 2 & 0 \\
-3 & -1 & 5
\end{bmatrix}
.
\]
\\
\end{example}

First with the Quick and Dirty method.
\\

Compute eigenvalues: $ \mydet(A-\lambda I)=0 $... get $\lambda=2$.

Pick a random vector: $ v = (1,5,3) $.

Multiply... $y = Tv - \lambda v = (1,0,1)$.

Multiply... $Ty - \lambda y = \0$.

Solve $Tx = \lambda x + v$...
no solutions (echelon form has a pivot at the last column).

To pick a vector outside the span, perform row reduction on $[ v, y, I ]_{3 \times 5}$... there are pivots on the first three rows, so we can take $z=(1,0,0)$.

Solve $Tx = \lambda x + z$...
no solutions (echelon form has a pivot at the last column).

Multiply... $w = Tz - \lambda z = (-3,0,-3)$.

Multiply again... $Tw - \lambda w = \0$.

We got four vectors, so \textbf{the method failed}.
\\

Let's see with the Guaranteed Method.
\\

Compute eigenvalues: $ \mydet(A-\lambda I)=0 $... get $\lambda=2$.

Take $N=A-2I$.
Multiplying... $N^3 = \0$, so $U_0 =\C^3$, take the canonical basis.

We now compute $U_1$.

Multiplying... 
$y_1 = Ne_1 = (-3,0,-3)$,
$y_2 = Ne_2 = (-1,0,-1)$,
$y_3 = Ne_3 = (3,0,3)$.

Performing row reduction on $ [y_1,y_2,y_3] $... we get pivot only at the first column, so $\{y_1\}$ is a basis for $U_1$.

We now compute $U_2$.

$Ny_1 = \0$, so $U_3 = U_2 = \{\0\}$.

We now build the basis from top down:

For $k=3$, $\B_3 = \emptyset$.

For $k=2$, $\B_2 = \emptyset$.

For $k=1$: $\B_1 = \{y_1\}$.

For $k=0$:

Solve $Nx = y_1$... get a solution $z = (2,0,1)$.

Take $\tilde{\B}_0 = \{y_1,z\}$.

Since $\{y_1,z,e_1,e_2,e_3\}$ span $U_0$, we perform row reduction on this $3 \times 5$ matrix... get pivots on columns 1 and 2 (as expected) as well as 4.
So take $w=e_2$.

Multiplying... $Nw = (-1,0,-1)$.

Solving for $Nx = (-1,0,-1)$ with $x \in \myspan(z)$... we get $x = \frac{1}{3}z$.
To avoid fractions, we take $u = 3 (w-x) = (-2,3,-1)$.

So $\B_0 = \{y_1,z,u\}$.

We already know what the Jordan form is and how to write the basis.
Let's double-check:
\[
Q=
\begin{bmatrix}
-3 & 2 & -2 \\
0 & 0 & 3 \\
-3 & 1 & -1
\end{bmatrix}
\
\Longrightarrow
\
Q^{-1}AQ = 
\left[
\begin{array}{rr|r}
2 & 1 & 0 \\
0 & 2 & 0 \\
\hline
0 & 0 & 2
\end{array}
\right]
.
\]

\clearpage

\section*{Simpler method}

Produce some threads by picking vectors at random, then apply the \emph{stretch and reduce} algorithm.
If necessary, add new threads to get a family that spans $G(\lambda,T)$.
\\

\hfil
See our other handout entitled \emph{Finding a Jordan basis for a nilpotent operator}.
\\

\begin{example}
Let us revisit the example where Quick and Dirty failed.
\end{example}

We got 4 vectors.
They form 2 closed threads $\A_1,\A_2$:
\[
(1,5,3) \mapsto (1,0,1) \mapsto \0
, \ \ \
(1,0,0) \mapsto (-3,0,-3) \mapsto \0
.
\]
Subtracting $-3 \A_1$ from $\A_2$ gives
\[
(1,5,3) \mapsto (1,0,1) \mapsto \0
, \ \ \
(4,15,9) \mapsto \0 \mapsto \0
.
\]

The threads are closed and the tips are L.I.
So regardless of how we got here, we found a Jordan basis!

We already know what the Jordan form is and how to write the basis.
Let's double-check:
\[
Q=
\begin{bmatrix}
1 & 1 & 4 \\
0 & 5 & 15 \\
1 & 3 & 9
\end{bmatrix}
\
\Longrightarrow
\
Q^{-1}AQ = 
\left[
\begin{array}{rr|r}
2 & 1 & 0 \\
0 & 2 & 0 \\
\hline
0 & 0 & 2
\end{array}
\right]
.
\]

\clearpage

\begin{example}
\[
[T] =
A=
\left[\begin{matrix}1 & 18 & -8 & -2 & -9\\-4 & 1 & 1 & -4 & 1\\-3 & -7 & 5 & -2 & 4\\-2 & -17 & 8 & 1 & 9\\-5 & 7 & -2 & -6 & -1\end{matrix}\right]
\]
\end{example}

Eigenvalues are given: 1 and 2.

Start with $\lambda=2$,
\[
N=
\left[\begin{matrix}-1 & 18 & -8 & -2 & -9\\-4 & -1 & 1 & -4 & 1\\-3 & -7 & 3 & -2 & 4\\-2 & -17 & 8 & -1 & 9\\-5 & 7 & -2 & -6 & -3\end{matrix}\right]
.
\]
Solving $N^5 x = \0$... we get $ \{(1,0,0,-1,0),(0,0,1,0,-1)\} $ as a basis $G(2,T)$.
We take the simple thread $ (1,0,0,-1,0) \mapsto (1,0,-1,-1,1) \mapsto \0 $ and we got a Jordan basis for $T$ restricted to $G(2,T)$.

Now with $\lambda=1$
\[
N=
\left[\begin{matrix}0 & 18 & -8 & -2 & -9\\-4 & 0 & 1 & -4 & 1\\-3 & -7 & 4 & -2 & 4\\-2 & -17 & 8 & 0 & 9\\-5 & 7 & -2 & -6 & -2\end{matrix}\right]
.
\]
Solving $N^5 x = \0$... we get $ \{( 1, 1, 0, 0, 2 ), ( 11, -1, 0, -8, 0 ), ( 3, 7, 16, 0, 0 )\} $ as a basis $G(1,T)$.
Computing each thread, we get first $( 1, 1, 0, 0, 2 ) \mapsto (0,-2,-2,-1,-2) \mapsto \0 $, and then
\[
(11,-1,0,-8,0)
\mapsto
(-2,-12,-10,-5,-14)
\mapsto
(0,4,4,2,4)
\mapsto
\0
.
\]
We can ignore the first thread, and the second thread alone provides a Jordan basis!

We already know what the Jordan form is and how to write the basis.
Let's double-check:
\[
Q=
\left[\begin{matrix}1 & 1 & 0 & -2 & 11\\0 & 0 & 4 & -12 & -1\\-1 & 0 & 4 & -10 & 0\\-1 & -1 & 2 & -5 & -8\\1 & 0 & 4 & -14 & 0\end{matrix}\right]
\
\Longrightarrow
\
Q^{-1}AQ = 
\left[\begin{array}{rr|rrr}2 & 1 & 0 & 0 & 0\\0 & 2 & 0 & 0 & 0\\\hline 0 & 0 & 1 & 1 & 0\\0 & 0 & 0 & 1 & 1\\0 & 0 & 0 & 0 & 1\end{array}\right]
.
\]

\end{document}